\usepackage[T1]{fontenc}
\usepackage[utf8]{inputenc}
\usepackage[brazilian]{babel}
\usepackage{epsfig}
\usepackage{subfigure}
\usepackage{amsfonts}
\usepackage{amsmath,mathrsfs}
\usepackage{amssymb}
\usepackage[thmmarks,amsmath]{ntheorem}%\usepackage{amsthm}
\usepackage{boxedminipage}
\usepackage{geometry}
\usepackage{theorem}
\usepackage{fancybox}
\usepackage{fancyhdr}
\usepackage{ifthen}
\usepackage{url}
\usepackage{afterpage}
\usepackage{color}
\usepackage{colortbl}
\usepackage{rotating}
\usepackage{makeidx}
\usepackage{epstopdf}
\usepackage{indentfirst}
\usepackage[pdfstartview=FitH]{hyperref}
\usepackage[table,xcdraw]{xcolor}
\usepackage{multirow}
\usepackage{pdfpages}
\usepackage{lastpage}
\usepackage{enumerate}

%pacote para usar o comando \widebar{}
\usepackage{mathabx}
%frames para notas
\usepackage[framemethod=default]{mdframed}


\usepackage[thinlines]{easytable}
%configuracao de legendas para figuras e graficos
\usepackage[labelfont=small,textfont={small,it}]{caption}

%secao para codigos
\usepackage[framed,numbered,autolinebreaks,useliterate]{mcode}



%----------------------------------------comandos para enunciar e enumerar enunciados
\newcommand{\dem}{\quad {\bf Demonstração:}\quad}
\newcommand{\fim}{\newline \begin{flushright} $\blacksquare$ \end{flushright}}
\newtheorem{defi}{Definição}[section]
\newtheorem{lema}{Lema}[section]
\newtheorem{conj}{Conjectura}[section]
\newtheorem{exem}{Exemplo}[section]
\newtheorem{axi}{Axioma}[section]
\newtheorem{prop}{Propriedade}[section]
\newtheorem{obs}{Observação}[section]
\newtheorem{teo}{Teorema}[section]
\newtheorem{corol}{Corolário}[section]
\newtheorem{res}{Resolução}[section]
%----------------------------------------------sí­mbolos
\newcommand{\R}{\mathbb{R}}%reais
\newcommand{\dotp}{\bullet}%produto escalar
\newcommand{\K}{\mathbb{K}}%corpo qualquer
\newcommand{\C}{\mathbb{C}}%complexos
\newcommand{\N}{\mathbb{N}}%naturais
\newcommand{\Z}{\mathbb{Z}}%inteiros
\newcommand{\Q}{\mathbb{Q}}%racionais
\newcommand{\U}{\mathcal{U}}%espaços vetoriais
\newcommand{\V}{\mathcal{V}}%espaços vetoriais
\newcommand{\W}{\mathcal{W}}%espaços vetoriais
\newcommand{\E}{\mathbb{E}}%espaços vetoriais
\newcommand{\HRule}{\rule{\linewidth}{0.5mm}}
\newcommand{\X}{$\bullet$}
\newcommand{\ignore}[1]{}

%-------------------------------------------------------------------------------------


\global\mdfdefinestyle{exampledefault}{%
	linecolor=lightgray,linewidth=1pt,%
	leftmargin=1cm,rightmargin=1cm,
}


%\usepackage[pdftex,a5paper,%
%pdftitle={The Ducks},%
%pdfauthor={Mother Goose},%
%colorlinks=true,%
%linkcolor=blue%
%]
\usepackage[framed,numbered]{config/mcode}
%\lstinputlisting{/path_to_mfile/yourmfile.m}
\hypersetup{colorlinks,%
citecolor=black,%
filecolor=black,%
linkcolor=black,%
urlcolor=black
}
\usepackage{array}
\usepackage{tabularx}
\usepackage{caption} 
%\captionsetup[table]{skip=10pt}
\newcolumntype{C}[1]{>{\centering\let\newline\\\arraybackslash\hspace{0pt}}m{#1}}
\newcolumntype{L}[1]{>{\raggedright\let\newline\\\arraybackslash\hspace{0pt}}m{#1}}
\newcolumntype{R}[1]{>{\raggedleft\let\newline\\\arraybackslash\hspace{0pt}}m{#1}}
\usepackage[font=small]{caption}
\usepackage{lastpage}
% Escolher um dos seguintes formatos:
\usepackage{config/ft_unb} % segue padrão de fonte Times
%\usepackage[num,abnt-etal-list=0]{config/abntex2/abntex2cite} % Citações pela ABNT
\usepackage[alf,abnt-etal-list=0]{config/abntex2/abntex2cite}

\newcommand{\citeC}[1]{[\citeonline{#1}]}
\renewcommand{\labelitemi}{$-$}


% Ambiente para inserir notas no meio do texto
\newenvironment{mymdframed}[1]{%
	\mdfsetup{%
		frametitle={\colorbox{white}{\,#1\,}},
		frametitleaboveskip=-\ht\strutbox,
		frametitlealignment=\raggedright
	}%
\begin{mdframed}[style=exampledefault]
}{\end{mdframed}}
