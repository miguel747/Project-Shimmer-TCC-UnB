\section{Códigos em MATLAB relacionados ao Método Proposto}

\textbf{Código MATLAB responsável pela realização do efeito shimmer numa amostra de Áudio}
	
	\begin{lstlisting}[caption={Code: "audio.m":Código Matlab para a realização da análise espectral do efeito shimmer em um sinal de áudio de 12bits/44100Hz},label={codigo01}]
		clc;
		clear all;
		info = audioinfo('guitar.wav')
		[y,Fs] = audioread('guitar.wav','native');
		voiceL = y(1:1000000,1);
		%voiceR = y(:,2);
		
		for j = 1:15
		voiceL = [voiceL, bitshift(voiceL(:,j),-1)]; % shift once to the right
		%   voiceR = [voiceR, bitshift(voiceR(:,j),-1)];
		end
		
		voice = [voiceL];%todas as taxas de bits por amostras de 16bits ate 1
		%voice = voice/max(max(abs(voice)));
		
		output12bits = 2\^4.*voice(:,5);%coluna 5 da esquerda 12bits e 
		%coluna 20 do audio direito tbm
		%12bits
		output10bits = 2\^6.*voice(:,7);
		output8bits = 2\^8.*voice(:,9);
		output4bits = 2\^12.*voice(:,13);
		
		audiowrite('guitar12bits.wav',output12bits,info.SampleRate);
		audiowrite('guitar10bits.wav',output10bits,info.SampleRate);
		audiowrite('guitar8bits.wav',output8bits,info.SampleRate);
		audiowrite('guitar4bits.wav',output4bits,info.SampleRate);
		
		info = audioinfo('guitar12bits.wav')
		info = audioinfo('guitar10bits.wav')
		info = audioinfo('guitar8bits.wav')
		info = audioinfo('guitar4bits.wav')
		
		
		%Amostrando os sinais no tempo
		figure(1)
		hold on
		t = linspace(0,1/info.SampleRate,length(voice));
		plot(t,voice(:,1));%16 bits
		%plot(t,output12bits);%12 bits
		%plot(t,output10bits);%10 bits
		plot(t,output8bits);%8 bits
		%plot(t,output4bits);%4 bits
		legend('sinal 16 bits','sinal 8 bits');
		hold off;
		
		
		% comparacao valores maximos
		Max12bits = max(2\^4.*voice(:,5))
		Max16bits = max(voice(:,1))
		
		%amostrando o sinal no dominio da freq
		output12bits = double(output12bits);
		Y1 =    fft(output12bits);
		N = Fs;
		transform = fft(output12bits,N)/N;
		magtransform = abs(transform)/abs(max(abs(transform)));
		num_bits = length(magtransform);
		plot([0:1/(num_bits/2-1):1],magtransform(1:num_bits/2))
		% faxis = linspace(-Fs/2,Fs/2,N);
		% figure()
		% plot(faxis,magtransform);
		xlabel('frequency(Hz)')
		
		%projetar o filtro
		[b,a] = butter(8,0.1,'low');
		H = freqz(b,a,floor(num_bits/2));
		hold on
		figure()
		plot([0:1/(num_bits/2 -1):1], abs(H),'r');
		hold off
		figure()
		output12bits_filtrado = filter(b,a,output12bits);
		plot(output12bits,'b')
		hold on
		plot(output12bits_filtrado,'r')
		%normalizando
		output12bits_filtrado = output12bits_filtrado/max(abs(output12bits_filtrado));
		legend('audio 12 bits','audio 12bits filtrado');
		audiowrite('guitar12bitsfiltrado.wav',output12bits_filtrado,info.SampleRate);
		
		hold off
		%oitavador
		guitar_oitavado = pitchShift(output12bits,1024,256,2);
		guitar_oitavado_filter = filter(b,a,guitar_oitavado);
		
		%pequeno delay
		leftout=output12bits;  % set up a new array, same size as old one
		
		N=100;  % delay amount N/44100 seconds
		
		for n=N+1:length(guitar_oitavado_filter)
		
		leftout(n)=output12bits(n)' + guitar_oitavado_filter(n-N);  % approximately 1/4 second echo
		end
		output12bits = double(output12bits);
		Y1 =    fft(output12bits);
		N = Fs;
		transform = fft(output12bits,N)/N;
		magtransform1 = abs(transform)/abs(max(abs(transform)));
		num_bits = length(magtransform1);
		plot([0:1/(num_bits/2-1):1],magtransform1(1:num_bits/2))
		% faxis = linspace(-Fs/2,Fs/2,N);
		% figure()
		% plot(faxis,magtransform);
		hold on
		leftout = double(leftout);
		Y1 =    fft(leftout);
		N = Fs;
		transform = fft(leftout,N)/N;
		magtransform2 = abs(transform)/abs(max(abs(transform)));
		num_bits = length(magtransform2);
		plot([0:1/(num_bits/2-1):1],magtransform2(1:num_bits/2))
		% faxis = linspace(-Fs/2,Fs/2,N);
		% figure()
		% plot(faxis,magtransform);
		xlabel('frequency(Hz)')
		figure()
		plot(leftout)
		hold on
		plot(output12bits)
		legend('sinal de 12 bits','sinal com pitchshift+delay')
		leftout = leftout/max(abs(leftout));
		output12bits = output12bits/max(abs(output12bits));
		
		audiowrite('shimmerA.wav',leftout,44100)
	\end{lstlisting}
	
	\textbf{Cálculo do Conversor DAC de 12 bits. (amostragem Linear)}
	Code: "mcp4725.c":
	\begin{lstlisting}
		#include <msp430.h> 
		#include <stdint.h>
		#include "mcp4725.h"
		#include "lib/lcd/lcd.h"
		#include "lib/dma/dma.h"
		#include "lib/port/port.h"
		#include "lib/clock/clock.h"
		#include "lib/adc12/adc12.h"
		#include "lib/timers/timer.h"
		#include "lib/serial/serial.h"
		
		#define MCP4725 0x62
		#define mcpON      0
		#define mcpOFF1K   1
		#define mcpOFF100K 2
		#define mcpOFF500K 3
		
		uint16_t adcResult;
		uint16_t data = 0, write;
		
		int main(void) {
		watchDogStop();
		portInit();
		
		lcdInit();
		lcdClear();
		
		clockInit();
		clockSetDCO(1000000);
		clockSelect(DCO, SMCLK);
		clockSelect(DCO,  MCLK);
		
		adc12Init();
		portRoute2Perif(P6,0);
		
		//timerSetup(B0,ACLK,UP,3276,1000);
		TB0CTL   = TBSSEL__ACLK    |            // Select ACLK as clock source
		MC__UP          |            // Setup but do not count
		TBCLR;                       // Clear timer
		
		TB0CCR0  = 327;                        // Convert every 100ms
		TB0CCR1  = 100;                        // This can be anything
		TB0CCTL0 = CCIE;
		TB0CCTL1 = OUTMOD_3;                    // Set/reset
		
		dmaEnable(0);
		dmaTrgr(0,DMA_ADC12IFGx);
		dmaAddr(0,&ADC12MEM0,DMA_FIXED,&adcResult,DMA_FIXED);
		dmaMode(0,DMA_RPT_SINGLE_TRANSFER);
		dmaSize(0,1);
		
		serialInit(I2C);
		
		__enable_interrupt();
		
		//mcpWrite(0x2FF);
		
		volatile uint16_t recData[4096];
		uint16_t index = 4096;
		
		while(index) {
		while(!write);
		recData[--index] = adcResult;
		mcpWrite(index);
		write = 0;
		}
		while(1);
		}
		
		void mcpWrite(uint16_t data) {
		uint8_t vector[2];
		vector[0] = (data >> 8) & 0x0F;
		vector[1] = (data     ) & 0xFF;
		serialI2CSendData(MCP4725, vector, 2);
		}
		
		#pragma vector=TIMER0_B0_VECTOR
		__interrupt void isr_tb0_ccr0 () {
		write = 1;
		adcDisable();
		adcEnable();
		}
			
	\end{lstlisting}
	
\textbf{Código Auxiliar responsável pelo Efeito Shimmer. Há duas funções aninhadas necessárias que serão inseridas no final do código.}
	
	\begin{lstlisting}[caption={Code: "pitchshifting.m":Código Matlab da função auxiliar responsável pelo pitch-shifter},label={codigo02}]
		
		
		function [outputVector] = pitchShift(inputVector, windowSize, hopSize, step)
		
		%% Parameters
		
		% Window size
		winSize = windowSize;
		% Space between windows
		hop = hopSize;
		% Pitch scaling factor
		alpha = 2^(step/12);
		
		% Intermediate constants
		hopOut = round(alpha*hop);
		
		% Hanning window for overlap-add
		wn = hann(winSize*2+1);
		wn = wn(2:2:end);
		
		%% Source file
		
		x = inputVector;
		
		% Rotate if needed
		if size(x,1) < size(x,2)
		
		x = transpose(x);
		
		end
		
		x = [zeros(hop*3,1) ; x];
		
		%% Initialization
		
		% Create a frame matrix for the current input
		[y,numberFramesInput] = createFrames(x,hop,winSize);
		
		% Create a frame matrix to receive processed frames
		numberFramesOutput = numberFramesInput;
		outputy = zeros(numberFramesOutput,winSize);
		
		% Initialize cumulative phase
		phaseCumulative = 0;
		
		% Initialize previous frame phase
		previousPhase = 0;
		
		for index=1:numberFramesInput
		
		%% Analysis
		
		% Get current frame to be processed
		currentFrame = y(index,:);
		
		% Window the frame
		currentFrameWindowed = currentFrame .* wn' / sqrt(((winSize/hop)/2));
		
		% Get the FFT
		currentFrameWindowedFFT = fft(currentFrameWindowed);
		
		% Get the magnitude
		magFrame = abs(currentFrameWindowedFFT);
		
		% Get the angle
		phaseFrame = angle(currentFrameWindowedFFT);
		
		%% Processing    
		
		% Get the phase difference
		deltaPhi = phaseFrame - previousPhase;
		previousPhase = phaseFrame;
		
		% Remove the expected phase difference
		deltaPhiPrime = deltaPhi - hop * 2*pi*(0:(winSize-1))/winSize;
		
		% Map to -pi/pi range
		deltaPhiPrimeMod = mod(deltaPhiPrime+pi, 2*pi) - pi;
		
		% Get the true frequency
		trueFreq = 2*pi*(0:(winSize-1))/winSize + deltaPhiPrimeMod/hop;
		
		% Get the final phase
		phaseCumulative = phaseCumulative + hopOut * trueFreq;    
		
		% Remove the 60 Hz noise. This is not done for now but could be
		% achieved by setting some bins to zero.
		
		%% Synthesis    
		
		% Get the magnitude
		outputMag = magFrame;
		
		% Produce output frame
		outputFrame = real(ifft(outputMag .* exp(j*phaseCumulative)));
		
		% Save frame that has been processed
		outputy(index,:) = outputFrame .* wn' / sqrt(((winSize/hopOut)/2));
		
		end
		
		%% Finalize
		
		% Overlap add in a vector
		outputTimeStretched = fusionFrames(outputy,hopOut);
		
		% Resample with linearinterpolation
		outputTime = interp1((0:(length(outputTimeStretched)-1)),
		outputTimeStretched,
		(0:alpha:(length(outputTimeStretched)-1)),'linear');
		
		% Return the result
		outputVector = outputTime;
		
		return
		
		function [vectorFrames,numberSlices] = createFrames(x,hop,windowSize)
		
		% Find the max number of slices that can be obtained
		numberSlices = floor((length(x)-windowSize)/hop);
		
		% Truncate if needed to get only a integer number of hop
		x = x(1:(numberSlices*hop+windowSize));
		
		% Create a matrix with time slices
		vectorFrames = zeros(floor(length(x)/hop),windowSize);
		
		% Fill the matrix
		for index = 1:numberSlices
		
		indexTimeStart = (index-1)*hop + 1;
		indexTimeEnd = (index-1)*hop + windowSize;
		
		vectorFrames(index,:) = x(indexTimeStart: indexTimeEnd);
		
		end
		
		return

	
		function vectorTime = fusionFrames(framesMatrix, hop)
		
		sizeMatrix = size(framesMatrix);
		
		% Get the number of frames
		numberFrames = sizeMatrix(1);
		
		% Get the size of each frame
		sizeFrames = sizeMatrix(2);
		
		% Define an empty vector to receive result
		vectorTime = zeros(numberFrames*hop-hop+sizeFrames,1);
		
		timeIndex = 1;
		
		% Loop for each frame and overlap-add
		for index=1:numberFrames
		
		vectorTime(timeIndex:timeIndex+sizeFrames-1) = 
		vectorTime(timeIndex:timeIndex+sizeFrames-1) + framesMatrix(index,:)';
		
		timeIndex = timeIndex + hop;
		
		end
		
		return
	\end{lstlisting}

