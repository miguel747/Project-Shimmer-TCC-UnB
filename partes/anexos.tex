Code: "audio.m": Código MATLAB responsável pela realização do efeito shimmer numa amostra de Áudio:
\begin{lstlisting}
	clc;
	clear all;
	info = audioinfo('guitar.wav')
	[y,Fs] = audioread('guitar.wav','native');
	voiceL = y(1:1000000,1);
	%voiceR = y(:,2);
	
	for j = 1:15
	voiceL = [voiceL, bitshift(voiceL(:,j),-1)]; % shift once to the right
	%   voiceR = [voiceR, bitshift(voiceR(:,j),-1)];
	end
	
	voice = [voiceL];%todas as taxas de bits por amostras de 16bits ate 1
	%voice = voice/max(max(abs(voice)));
	
	output12bits = 2\^4.*voice(:,5);%coluna 5 da esquerda 12bits e 
	%coluna 20 do audio direito tbm
	%12bits
	output10bits = 2\^6.*voice(:,7);
	output8bits = 2\^8.*voice(:,9);
	output4bits = 2\^12.*voice(:,13);
	
	audiowrite('guitar12bits.wav',output12bits,info.SampleRate);
	audiowrite('guitar10bits.wav',output10bits,info.SampleRate);
	audiowrite('guitar8bits.wav',output8bits,info.SampleRate);
	audiowrite('guitar4bits.wav',output4bits,info.SampleRate);
	
	info = audioinfo('guitar12bits.wav')
	info = audioinfo('guitar10bits.wav')
	info = audioinfo('guitar8bits.wav')
	info = audioinfo('guitar4bits.wav')
	
	
	%Amostrando os sinais no tempo
	figure(1)
	hold on
	t = linspace(0,1/info.SampleRate,length(voice));
	plot(t,voice(:,1));%16 bits
	%plot(t,output12bits);%12 bits
	%plot(t,output10bits);%10 bits
	plot(t,output8bits);%8 bits
	%plot(t,output4bits);%4 bits
	legend('sinal 16 bits','sinal 8 bits');
	hold off;
	
	
	% comparacao valores maximos
	Max12bits = max(2\^4.*voice(:,5))
	Max16bits = max(voice(:,1))
	
	%amostrando o sinal no dominio da freq
	output12bits = double(output12bits);
	Y1 =    fft(output12bits);
	N = Fs;
	transform = fft(output12bits,N)/N;
	magtransform = abs(transform)/abs(max(abs(transform)));
	num_bits = length(magtransform);
	plot([0:1/(num_bits/2-1):1],magtransform(1:num_bits/2))
	% faxis = linspace(-Fs/2,Fs/2,N);
	% figure()
	% plot(faxis,magtransform);
	xlabel('frequency(Hz)')
	
	%projetar o filtro
	[b,a] = butter(8,0.1,'low');
	H = freqz(b,a,floor(num_bits/2));
	hold on
	figure()
	plot([0:1/(num_bits/2 -1):1], abs(H),'r');
	hold off
	figure()
	output12bits_filtrado = filter(b,a,output12bits);
	plot(output12bits,'b')
	hold on
	plot(output12bits_filtrado,'r')
	%normalizando
	output12bits_filtrado = output12bits_filtrado/max(abs(output12bits_filtrado));
	legend('audio 12 bits','audio 12bits filtrado');
	audiowrite('guitar12bitsfiltrado.wav',output12bits_filtrado,info.SampleRate);
	
	hold off
	%oitavador
	guitar_oitavado = pitchShift(output12bits,1024,256,2);
	guitar_oitavado_filter = filter(b,a,guitar_oitavado);
	
	%pequeno delay
	leftout=output12bits;  % set up a new array, same size as old one
	
	N=100;  % delay amount N/44100 seconds
	
	for n=N+1:length(guitar_oitavado_filter)
	
	leftout(n)=output12bits(n)' + guitar_oitavado_filter(n-N);  % approximately 1/4 second echo
	end
	output12bits = double(output12bits);
	Y1 =    fft(output12bits);
	N = Fs;
	transform = fft(output12bits,N)/N;
	magtransform1 = abs(transform)/abs(max(abs(transform)));
	num_bits = length(magtransform1);
	plot([0:1/(num_bits/2-1):1],magtransform1(1:num_bits/2))
	% faxis = linspace(-Fs/2,Fs/2,N);
	% figure()
	% plot(faxis,magtransform);
	hold on
	leftout = double(leftout);
	Y1 =    fft(leftout);
	N = Fs;
	transform = fft(leftout,N)/N;
	magtransform2 = abs(transform)/abs(max(abs(transform)));
	num_bits = length(magtransform2);
	plot([0:1/(num_bits/2-1):1],magtransform2(1:num_bits/2))
	% faxis = linspace(-Fs/2,Fs/2,N);
	% figure()
	% plot(faxis,magtransform);
	xlabel('frequency(Hz)')
	figure()
	plot(leftout)
	hold on
	plot(output12bits)
	legend('sinal de 12 bits','sinal com pitchshift+delay')
	leftout = leftout/max(abs(leftout));
	output12bits = output12bits/max(abs(output12bits));
	
	audiowrite('shimmerA.wav',leftout,44100)
\end{lstlisting}

Code: "mcp4725.c" - Cálculo do Conversor DAC de 12 bits. (amostragem Linear):
\begin{lstlisting}
	#include <msp430.h> 
	#include <stdint.h>
	#include "mcp4725.h"
	#include "lib/lcd/lcd.h"
	#include "lib/dma/dma.h"
	#include "lib/port/port.h"
	#include "lib/clock/clock.h"
	#include "lib/adc12/adc12.h"
	#include "lib/timers/timer.h"
	#include "lib/serial/serial.h"
	
	#define MCP4725 0x62
	#define mcpON      0
	#define mcpOFF1K   1
	#define mcpOFF100K 2
	#define mcpOFF500K 3
	
	uint16_t adcResult;
	uint16_t data = 0, write;
	
	int main(void) {
	watchDogStop();
	portInit();
	
	lcdInit();
	lcdClear();
	
	clockInit();
	clockSetDCO(1000000);
	clockSelect(DCO, SMCLK);
	clockSelect(DCO,  MCLK);
	
	adc12Init();
	portRoute2Perif(P6,0);
	
	//timerSetup(B0,ACLK,UP,3276,1000);
	TB0CTL   = TBSSEL__ACLK    |            // Select ACLK as clock source
	MC__UP          |            // Setup but do not count
	TBCLR;                       // Clear timer
	
	TB0CCR0  = 327;                        // Convert every 100ms
	TB0CCR1  = 100;                        // This can be anything
	TB0CCTL0 = CCIE;
	TB0CCTL1 = OUTMOD_3;                    // Set/reset
	
	dmaEnable(0);
	dmaTrgr(0,DMA_ADC12IFGx);
	dmaAddr(0,&ADC12MEM0,DMA_FIXED,&adcResult,DMA_FIXED);
	dmaMode(0,DMA_RPT_SINGLE_TRANSFER);
	dmaSize(0,1);
	
	serialInit(I2C);
	
	__enable_interrupt();
	
	//mcpWrite(0x2FF);
	
	volatile uint16_t recData[4096];
	uint16_t index = 4096;
	
	while(index) {
	while(!write);
	recData[--index] = adcResult;
	mcpWrite(index);
	write = 0;
	}
	while(1);
	}
	
	void mcpWrite(uint16_t data) {
	uint8_t vector[2];
	vector[0] = (data >> 8) & 0x0F;
	vector[1] = (data     ) & 0xFF;
	serialI2CSendData(MCP4725, vector, 2);
	}
	
	#pragma vector=TIMER0_B0_VECTOR
	__interrupt void isr_tb0_ccr0 () {
	write = 1;
	adcDisable();
	adcEnable();
	}
	
\end{lstlisting}
