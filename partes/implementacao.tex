\chapter{Implementação do Projeto}

\section{Bloco 0 - Conversão A/D no MSP430}
	
	Antes do sinal chegar ao microcontrolador para o correto processamento do sinal de áudio, será necessário, conforme já relatamos, amostrar esse sinal através de valores corretamente quantizados. Para isso, conforme abordado na seção \ref{secao-conv-analogica-digital} deste trabalho, é necessário utilizar um Conversor Analógico Digital - ADC.
	
	O modelo do MSP 430 em questão (MSP430F5529LP) possui um ADC interno de 12 (doze) bits (ADC12). Ou seja, é possível representar em níveis de pulso até 4096 níveis. Algumas características podemos listar pois será objeto de avaliação dentro do projeto \cite{Davies2008}.
	
	\begin{itemize}
		\item Resolução de 12 bits monotônica, sem perdas de código;
		\item Velocidade nominal de até 200.000 amostras por segundo (200 Ksps), utilizando a técnica de aproximação sucessivas (SAR);
		\item Operação de com diversas referências internas de tensões: 1.5V, 2.0V, ou 2.5V com consumo típico de aproximadamente $250\mu A$ quando em operação;
		\item Canais de entrada exclusivos para sensor de temperatura interno, tensão de alimentação e tensões de referências externas;
		\item 16 memórias de conversão com controle independente de cada uma, inclusive com a capacidade de especificar o canal de entrada e referência;
		\item Fonte de \textit{clock} selecionáveis por softwares;
	\end{itemize}

	O coração de funcionamento desse módulo do microcontrolador consiste basicamente no seguinte:
	
		\begin{enumerate}[(i)]
			\item O processador usa dois níveis de tensões selecionáveis: $V_{R^{+}} $ e $ V_{R^{-}} $ afim de determinar o valor mínimo e máximo do conversor;
			\item Uma saída digital ($ N_{ADC} $) é setado no nível máximo ($ 4095 = 0FFFh $) quando o valor de entrada for igual ou até mesmo maior que $ V_R^+ $. De modo semelhante o valor digital ($ N_{ADC} $) será zero quando o valor de entrada for igual ou menor que $ V_{R^-} $.
			\item A equação básica da conversão é:
			
			\begin{equation}
				\label{eq-adc12-formula}
				N_{ADC} = 4095.\frac{V_{in}-V_R^-}{V_R^+ - V_R^-}
				\qquad
				V_{in}: \text{Tensão de entrada}.
			\end{equation}
			\item Em especial, o módulo ADC12\_A é configurado por dois registros de controle: o ADC12CTL0 e ADC12CTL1.
			\item  O modo de funcionamento do conversor (conversão simples ou sequência de canais) pode ser configurado pelos \textit{bits} CONSEQx (registrador ADC12CTL1);
			\item Após isso, seleciona-se o endereço inicial da memória de conversão, pelos \textit{bits} CSTARTADDx (registrador ADC12CTL1);
			\item Por fim, liga-se o conversor (\textit{bit} ADC12CTL0:ADC12ON=1) e habilitam-se as conversões (\textit{bit} ADC12CTL0:ENC=1).7
		\end{enumerate}
	
	
\section{Bloco 1 - Projetando um Filtro FIR}
	\subsection{Comunicação Serial}
		
\section{Bloco 2 - Implementando o \textit{Pitch-Shifter}}

\section{Bloco 3 - Delay Time}