\chapter{Metodologia e Ferramentas}


\section{Sinais de Tempo Discreto e Transformada Discreta de \textit{Fourier}}

\section{Filtros Digitais}

	\subsection{Conceitos Iniciais}
	
	Os filtros digitais não contém uma implementação física em si, diferentemente dos filtros analógicos constituídos, geralmente, de associação de resistores e capacitores. Eles são construídos através de algoritmos.
	
	Para que isso possa ocorrer é necessário que o sinal de áudio (analógico) seja devidamente convertido em um sinal digital. esse sinal portanto convolui por um algoritmo de filtro adequado.
	
	De maneira geral, o projeto de um filtro consite em obter os coeficientes para os filtros. Isso ~e realizado através de uma equação chamada de equação das diferenças. O processo pode ser simplesmente realizado pela equação:
	
	\begin{equation}
		Saida = \sum_{1}^{n} Coeficiente_n do filtro * Amostra_n
	\end{equation}
	
	
	\subsection{Filtros IIR}
	\subsection{Filtros FIR}
	\subsection{Filtros Adaptativos}

\section{Ferramentas Computacionais}